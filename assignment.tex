\documentclass[titlepage]{article}

\usepackage{polski}
\usepackage[utf8]{inputenc}
\usepackage{chemformula}
\usepackage{amsfonts}

\title{
	\Large{Modelowanie Matematyczne (zadanie)}
	\\
	\normalsize{Wojskowa Akademia Techniczna im. Jarosława Dąbrowskiego}
	\\
	\normalsize{prowadzący: mgr inż. Michał Kapałka}
	}

\author{Artur M. Wolff (grupa nr H7X2S1)}

\date{\today}

\begin{document}

\maketitle

\section{Treść problemu}
Warmińsko-mazurska kopalnia paliw kopalnianych \textit{Albrecht} wydobywa 3 rodzaje węgla.
Węgiel typu I jest najbardziej opłacalny do wydobycia, ale najmniej ekologiczny.
Węgiel typu II jest mniej opłacalny do wydobycia, ale bardziej ekologiczny.
Węgiel typu III jest najmniej opłacalny do wydobycia, ale najbardziej ekologiczny.
Dyrektywy Unii Europejskiej mówią o tym, że w pełni działająca kopalnia powinna wydobywać co najwyżej 25Mt węgla typu I, co najmniej 50Mt węgla typu II i co najmniej 25Mt węgla typu III miesięcznie.
Wydajność kopalni \textit{Albrecht} to średnio 133Mt miesięcznie.
Za każdy 1\% nadprodukcji węgla typu II kopalnia otrzyma \textit{dopłatę klimatyczną} w wysokości 5\% przychodu i 10\% przychodu w przypadku węgla typu III.
Jednak, w wyniku nieprzewidzianych okoliczności, może zdarzyć się tak, że podczas kopania węgla wykopany zostanie naturalnie występujący radioaktywny metal, izotop polonu \ch{^{210}Po}.
W takim wypadku, Polska Agencja Atomistyki zamyka kopalnię na miesiąc, a wykopany dotychczas węgiel zostaje przewieziony na wysypisko materiałów promieniotwórczych.
Wymienione powyżej zdarzenie to proces stochastyczny i nie da się go przewidzieć, ale jest możliwość zasymulowania go używając generatora liczb pseudolosowych (wartości boolowskich).
Poza tym, koszty związane z sezonowaniem (składowaniem i przechowywaniem) nadwyżek produkcji ponosi kopalnia.
Jakie powinny być proporcje wydobycia każdego z typów węgla w miesiącu, biorąc pod uwagę zmienny popyt na węgiel, tak aby zysk zakładu był jak największy?
Czy da się znaleźć rozwiązanie optymalne bez wprowadzania pojęcia ryzyka?

\section{Lista danych}
$x_1$ -- koszt wydobycia 1 tony węgla typu I, $x_1 \in \mathbb{N_+}$ \\
$x_2$ -- koszt wydobycia 1 tony węgla typu II, $x_2 \in \mathbb{N_+}$ \\
$x_3$ -- koszt wydobycia 1 tony węgla typu III, $x_3 \in \mathbb{N_+}$ \\
$c_1$ -- cena sprzedaży węgla typu I (1 tony), $c_1 \in \mathbb{N_+}$ \\
$c_2$ -- cena sprzedaży węgla typu II (1 tony), $c_2 \in \mathbb{N_+}$ \\
$c_3$ -- cena sprzedaży węgla typu III (1 tony), $c_3 \in \mathbb{N_+}$ \\
$u_1$ -- maksymalny udział węgla typu I w ogóle wydobycia, $u_1 \in \mathbb{N_+}$ \\
$u_2$ -- minimalny udział węgla typu II w ogóle wydobycia, $u_2 \in \mathbb{N_+}$ \\
$u_3$ -- minimalny udział węgla typu III w ogóle wydobycia, $u_3 \in \mathbb{N_+}$ \\
$k_2$ -- dopłata klimatyczna za nadprodukcję 1\% węgla typu II, $k_2 \in \mathbb{N_+}$ \\
$k_3$ -- dopłata klimatyczna za nadprodukcję 1\% węgla typu III, $k_3 \in \mathbb{N_+}$ \\
$p_1$ -- popyt na węgiel typu I, $p_1 \in \mathbb{N}$ \\
$p_2$ -- popyt na węgiel typu II, $p_2 \in \mathbb{N}$ \\
$p_3$ -- popyt na węgiel typu III, $p_3 \in \mathbb{N}$ \\
$s$ -- koszt sezonowania węgla, $s \in \mathbb{N_+}$ \\
$w$ -- wydajność kopalni, $w \in \mathbb{N_+}$ \\
$b$ -- wynik generatora liczb pseudolosowych (PRNG)*, $b \in \{ 0, 1 \}$

\subsection{*Generator liczb pseudolosowych}
PRNG dla zadania zdefiniowany jest w następujący sposób:
$$x_{n + 1} = (x_n)^2 \ mod \ M$$
gdzie $x_n$ to kolejne stany generatora, a $M$ to iloczyn dwóch dużych liczb pierwszych $p$ i $q$ dających w dzieleniu przez 4 resztę 3 (dzięki czemu każda reszta kwadratowa modulo $p$ ma jeden pierwiastek kwadratowy,
który także jest resztą kwadratową), i mających możliwie mały $\operatorname{NWD}(\phi(p - 1), \phi(q - 1))$, a $\phi$ jest funkcją Eulera (co zapewnia długi cykl). Wynikiem generatora jest ostatni bit $x_n$.

\section{Lista zmiennych decyzyjnych}
$l_1$ -- liczba wyprodukowanych ton węgla typu I, $l_1 \in \mathbb{N}$ \\
$l_2$ -- liczba wyprodukowanych ton węgla typu II, $l_2 \in \mathbb{N_+}$ \\
$l_3$ -- liczba wyprodukowanych ton węgla typu III, $l_3 \in \mathbb{N_+}$

\section{Lista wskaźników}
$z$ -- zysk kopalni, $z \in \mathbb{Z}$

\section{Model optymalizacyjny}
\begin{equation}
	a = \left< x_1, x_2, x_3, c_1, c_2, c_3, u_1, u_2, u_3, k_1, k_2, p_1, p_2, p_3, s, w, b \right>
\end{equation}

\begin{equation}
	\mathbb{A} = \{ a \in \mathbb{N_+}^{13} \times \mathbb{N}^3 \times \{ 0, 1 \} \}
\end{equation}

\begin{equation}
	x = \left< l_1, l_2, l_3 \right>
\end{equation}

\begin{equation}
	\Omega(a) = \{ x \in \mathbb{N} \times \mathbb{N_+}^2 : l_1 + l_2 + l_2 \leq w \land l_1 \leq u_1 \land l_2 \geq u_2 \land l_3 \geq u_3 \}
\end{equation}

\begin{equation}
	k = \left< z \right>
\end{equation}

\begin{equation}
	\begin{aligned}
		\operatorname{K}(a, x) ={} & \{ k \in \mathbb{Z} : k = f(x) = b \cdot \sum_{i = 1}^3 c_i \cdot \min{(p_i, l_i)} \ -         \\
		                           & + \sum_{i = 1}^3 x_i \cdot l_i - s \cdot \sum_{i = 1}^3 \max{(0, l_i - p_i)} \ +               \\
		                           & + k_2 \cdot (\frac{100 \cdot l_2}{u_2} - 100) + k_3 \cdot (\frac{100 \cdot l_3}{u_3} - 100) \} 
	\end{aligned}
\end{equation}

\section{Zadanie optymalizacyjne}
Dla danych $a \in \mathbb{A}$, wyznaczyć $x^* \in \Omega(a)$, takie, że $f(x^*) = \max_{x \in \Omega(a)}{f(x)}$.

\end{document}
